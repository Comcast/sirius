%% Header, styling stuff

\documentstyle[11pt]{article}
\setlength{\topmargin}{-.5in}
\setlength{\textheight}{9in}
\setlength{\oddsidemargin}{.125in}
\setlength{\textwidth}{6.25in}


\begin{document}

%%----------------
%% Title
%%----------------
\title{Sirius Detailed Design}
\author{Platform \& APIs Team}
\maketitle

This document is intended document Sirius’ design, so future developers can understand the system at a reasonable level, and so that we can record important design decisions somewhere in prose.  

This document is being written along with the system, it’s descriptive of the Sirius we are building, rather than prescriptive of the Sirius we will build.  It is not intended to be used as a specification, and does not contain enough detail to be used as such.

%%----------------------
%% Theory of Operation
%%----------------------
\section{Theory of Operation}
TBD


%%---------------------
%% Design Overview
%%---------------------
\section{Design Overview}
TBD


%%-------------------------
%% Write Ahead Log Format
%%-------------------------
\section{Write Ahead Log Format}

%%------------------------------
%% Write Ahead Log Format:
%%    Principles of Some Import
%%------------------------------
\subsection{Principles of Some Import}
The Sirius write-ahead log is designed with the following principles in mind, in rough priority order:

\begin{itemize}
\item It must be possible to reconstruct the entire state of a system built on top of Sirius by replaying the write-ahead log.
\item The write-ahead log should be easily compactable, so it can serve as our persistent storage format for the state of a system built on Sirius, rather than using a separate snapshot mechanism.
\item The log should be as human readable as it can be, or, readability by a human is more important than squeezing as much information into a given number of bytes as possible.
\end{itemize}


%%------------------------------
%% Write Ahead Log Format:
%%    Design Overview
%%------------------------------
\subsection{Design Overview}

The log file is plain text, UTF-8 encoded.  Each entry begins with a checksum, and is terminated by a line feed (LF/'\textbackslash n'/U+000A). The checksum covers all data from the byte after the checksum to the line feed, inclusive.

Fields in an individual log entry are pipe (U+007C) delineated. Pipe characters and all whitespace\footnote{See http://docs.oracle.com/javase/6/docs/api/java/lang/Character.html\#isWhitespace(char)} are disallowed as parts of the log elements.  For convenience, a pipe character is placed immediately after the checksum.

An typical log entry would look something like:
\begin{center}
\em Checksum\textbar Action Type\textbar Key\textbar Sequence\textbar Timestamp\textbar Payload\textbackslash n
\end{center}


%%------------------------------
%% Write Ahead Log Format:
%%    Field Definitions
%%------------------------------
\subsection{Field Definitions}

\subsubsection{Checksum}
Base64 encoded MD5 hash of the UTF-8 byte array that represents the rest of the log entry\footnote{It is worth mentioning, this is exactly 24 characters in length}, including the pipes and terminating new line.

\subsubsection{Action Type}
A String that represents the type of action to take for a given log entry.  For the most part, these will mirror HTTP semantics.

\begin{center}
\begin{tabular}{|l|p{3.5in}|}
\hline
\multicolumn{2}{|c|}{Action  Types}\\ \hline
\textbf{PUT} & An HTTP PUT\\ \hline
\textbf{DELETE} & An HTTP DELETE\\ \hline
\textbf{CONDITIONAL\_PUT} & A conditional HTTP PUT, which may or may not have succeeded.   Since the log is write-ahead, we won’t know if the PUT will be successful at log writing time.\\ \hline
\textbf{CONDITIONAL\_DELETE} & Same as above, but for HTTP DELETE.\\ \hline
\textbf{COMPACTION\_HINT} & A hint that tells the compaction algorithm whether a particular conditional put or delete that comes earlier in the log was successful.\\ \hline
\end{tabular}
\end{center}

\subsubsection{Key}
A key that’s used to determine what sort of action should be taken.  This is going to be something like a partial URL, so something like /videos/1234.

For instance, a PUT to a key of /videos/1234 might add a video represented by the payload to a map, along with adding it to some secondary indices.  

Note that pipes and various whitespace are not allowed in the key, though they may be allowed in URIs.

\subsubsection{Sequence}
Strictly increasing sequence number that defines a total order of requests in the log, across all nodes in the system.  This is a String representation of a twos-complement \textbf{signed}\footnote{Because Java that’s why.} 64 bit integer, and must not fall outside the range representable in that scheme.

\subsubsection{Timestamp}
UTC timestamp in some ISO 8601 compliant format.

\subsubsection{Payload}
Base64 encoded binary payload for the log entry.  This is probably some representation of the data needed to do a PUT to a given resource.


%%------------------------------
%% Write Ahead Log Format:
%%    Compaction Algorithm
%%------------------------------
\subsection{Compaction Algorithm}
TBD


%%------------------------------
%% Cluster Membership:
%%    Join Algorithm
%%------------------------------
\section{Cluster Membership}
\subsection{Caveats}
For now we are ignoring several facts of life.  We are assuming messages sent between nodes always are received in a timely manor and are never lost.  Also that Nodes never disappear or leave unexpectedly.
\subsection{Cluster Definition}
Let's define $i$ as a node's unique identifier and $M$ as a non-empty set of $i$ representing a list of the cluster's members.

\begin{itemize}
\item Axiom (1) - Cluster $C$ has at least one node.  $C \not= \emptyset$
\item Axiom (2) - A node $N$ is defined as $\{i,M\}$ where $i$ is a unique id, and $M$ is a set of id's of all  node's in the cluster that $N$ is a member of. 
\end{itemize}

\subsection{Joining a Cluster}
if $A$ and $B$ are nodes in two separate clusters who do not intersect, $A join B$ works like this.


\begin{enumerate}
\item $\forall M \epsilon B$ , set $M$  to $M\epsilon A \cup M\epsilon B$
\item set $M \epsilon A$ to $M\epsilon B$

\end{enumerate}




Here is some pseudo-code that describes how a node might handle the joins:


\begin{verbatim}
     class Node()
          var id = uniqueId()
          var members = {self().id}
          for-ever
               switch receive()
                    case {"join_cluster", nodeToJoin}
                         var newMembers := nodeToJoin.send("join",node,self().members)
                         self().members := newMembers
                    case {"join",newMembers}
                         for-all member in members: self().send("add_members", newMembers)
                         members ++ newMembers
                         sender().send("add_members",members)
                    case {"add_members", newMembers }
                         members ++  newMembers
          end-for-ever
\end{verbatim}



\end{document}
